\chapter{Measure Theory and Fractal Dimension}
\label{chap:measure}

Measure theory provides a powerful framework within which the notion of
dimension can be defined. This is done by analysing the measure of a given
set with respect to a class of measures called Hausdorff measures. The
construction of these measures is challenging, and determining the measures
of sets under them even more so, but the result of our effort is a deep and
beautiful theory of measure and dimension that justifies the difficulties
faced in the construction.

We begin with some preliminary measure-theoretic definitions and ideas. Let
$\mathbb{X}$ be a set. A $\sigma-algebra$ $\mathcal{M}$ of subsets of
$\mathbb{X}$ is a collection having the following properties:
   
\begin{enumerate} 
  \item $\mathbb{X} \in \mathcal{M}$

  \item If $A \in \mathcal{M}$, then ${A}^c \in \mathcal{M}$.

  \item If $\{A_i\}^\infty_{i=1}$ is a countable collection of sets in
    $\mathcal{M}$, then $\bigcup\limits_{i=1}^{\infty}A_i \in \mathcal{M}$.
\end{enumerate}

Other properties can be deduced from the above. For instance, by 1 and 2,
$\emptyset \in \mathcal{M}$; by 2, 3, and De Morgan's laws, if
$\{A_i\}^\infty_{i=1}$ is a countable collection of sets in $\mathcal{M}$,
then $\bigcap\limits_{i=1}^{\infty}A_i \in \mathcal{M}$. Elements of
$\mathcal{M}$ are called $measurable$. In a topological space, one can define
a special $\sigma-algebra$ called the Borel $\sigma-algebra$. It is defined
to be the smallest $\sigma-algebra$ containing all open sets in the
topological space. Since every metric space is a topological space, every
metric space gives rise to a unique Borel $\sigma-algebra$ $\mathcal{B}$.

A measure on $\mathcal{M}$ is a set function $\mu:\mathcal{M}\rightarrow\mathbb{R}$ such that:
  
\begin{enumerate}
  \item{1.} $\mu(\emptyset)=0$

  \item{(monotonicity)} If $A,B \in \mathcal{M}$ with $A \subset B$, then
    $\mu(A)\leq\mu(B)$.

  \item{(countable additivity)} If $\{A_i\}^\infty_{i=1}$ is a countable
    collection of pairwise disjoint sets in $\mathcal{M}$, then
    $\mu(\bigcup\limits_{i=1}^{\infty}A_i)=\sum\limits_{i=1}^{\infty}\mu(A_i)$
\end{enumerate} 

Measures are often constructed by restricting set functions called outer
measures, with domain $P^\mathbb{X}$, to a $\sigma-algebra$ of sets in
$P^\mathbb{X}$ satisfying a certain separation condition described below.

An outer measure is a function $\mu^*:P^\mathbb{X}\rightarrow\mathbb{R}$
satisfying conditions 1 and 2 of a measure, but with condition 3 replaced by
countable sub-additivity. That is, if $\{A_i\}^\infty_{i=1}$ is a countable
collection of sets in $P^\mathbb{X}$, then
$\mu^*(\bigcup\limits_{i=1}^{\infty}A_i)\leq\sum\limits_{i=1}^{\infty}\mu^*(A_i)$.
Now let $A \in P^\mathbb{X}$. If $\forall B \in P^\mathbb{X}$, we have
$\mu^*(B)=\mu^*(A \cap B)+\mu^*(A^c \cap B)$ (the separation condition
described above), then A is said to be measurable. The collection of
measurable sets in $P^\mathbb{X}$ form a $\sigma-algebra$, and the
restriction of $\mu^*$ to this $\sigma-algebra$ is a measure.

With the basic terminology of measure theory in place, we are ready to define
Hausdorff measures. Take $d \in \mathbb{R_+}$, and let $\mathbb{X}$ be a
metric space. For our purposes, X will be $\mathbb{R}^n$ for some $n \in
\mathbb{N}$. The d-Hausdorff outer measure, $m_d^*:P^\mathbb{X} \rightarrow
\mathbb{R}$, is defined for $A \in P^\mathbb{X}$ by: 

$m_d^*(A)=\lim_{\delta \to 0} \inf \{\sum\limits_{i=1}^{\infty}(diam(F_i))^d|A
\subset \bigcup\limits_{i=1}^{\infty}F_i$, $F_i \in P^\mathbb{X}$, and
$diam(F_i)<\delta\}$. 

Here, the $\inf$ is being taken over all countable coverings of A by sets $F_i
\in P^\mathbb{X}$ with $diam(F_i)<\delta$, and $diam(F)=\sup\{\|x-y\||x,y \in
F\}$, with $\| \|$ the Euclidean norm.

It turns out that $m_d^*$ is what is known as a metric outer measure.
Omitting the proof, this means that the Borel sets are measurable, and that
$m_d^*$ restricted to $\mathcal{B}$ is a measure. We denote this measure by
$m_d$, and note that $m_d:\mathcal{B} \rightarrow \mathbb{R}$. Note that for
$r \in \mathbb{R}$, $m_d^*(rA)=r^dm_d^*(A)$, as this fact will become
important later on.   

We are now ready to define the Hausdorff dimension of a set $A \subset
\mathbb{X}$. The Hausdorff dimension of A, $dim_H(A)$ is defined to be
$dim_H(A)=inf\{d \in \mathbb{R}| m_d^*(A)=0\}$, where the $\inf$ is taken
over all $d\geq 0$.

To see why we take the $\inf$ over this set, we first need the following theorem:

\newtheorem{theorem}{Theorem}

\begin{theorem} Suppose for some $d \in \mathbb{R}_+$ and $A \subset
  \mathbb{X}$ we have $m_d^*(A)<\infty$. Then, $\forall d'>d$, we have
  $m_{d'}^*(A)=0$.
\end{theorem}

\begin{proof} Take $d'>d$, $0<\delta<1$, and let $\{A_i\}_{i=1}^\infty$ be such
  that $\forall i$, $diam(A_i)<\delta$, $A \subset
  \bigcup\limits_{i=1}^{\infty}A_i$, and
  $\sum\limits_{i=1}^{\infty}(diam(A_i))^d < \infty$. This last part is
  possible since $m_d^*(A)<\infty$, so we can choose $\{A_i\}_{i=1}^\infty$ so
  that $\sum\limits_{i=1}^{\infty}(diam(A_i))^d < m_d^*(A) + 1 < \infty$ . 

  By definition, $\inf \{\sum\limits_{i=1}^{\infty}(diam(F_i))^{d'}|A \subset
  \bigcup\limits_{i=1}^{\infty}F_i$, $F_i \in P^\mathbb{X}$, and
  $diam(F_i)<\delta\}\leq \sum\limits_{i=1}^{\infty} (diam(A_i))^{d'}
  =\sum\limits_{i=1}^{\infty} (diam(A_i)^d)(diam(A_i))^{d'-d}\leq
  \sum\limits_{i=1}^{\infty}\delta^{d'-d}(diam(A_i))^d = \delta^{d'-d}
  \sum\limits_{i=1}^{\infty}(diam(A_i))^d$. 

  So, $\inf \{\sum\limits_{i=1}^{\infty}(diam(F_i))^{d'}|A \subset
  \bigcup\limits_{i=1}^{\infty}F_i$, $F_i \in P^\mathbb{X}$, and
  $diam(F_i)<\delta\}\leq \delta^{d'-d} \sum\limits_{i=1}^{\infty}(diam(A_i))^d$.
  Since $\delta <1$ and $d'-d>0$, taking the limit as $\delta \rightarrow 0$
  yields $m_{d'}^*(A)\leq 0$. Since $m_{d'}^*(A) \geq 0$, we have that
  $m_{d'}^*(A)=0$.
\end{proof}

By taking the contrapositive of Theorem 1, we see:

\begin{theorem} Suppose for some $d \in \mathbb{R}_+$ and some $A \subset
  \mathbb{X}$ we have that $m_d^*(A)>0$. Then, $\forall d'<d$, we have
  $m_{d'}^*(A)=\infty$.  
\end{theorem}

  Now consider a unit cube $C$ in $\mathbb{R}^d$. If we divide the length of
  the side by $2^k$, we can construct a cover consisting of $2^{dk}$ sub-cubes,
  each of diameter $2^{-k}\sqrt{d}$. If denote this collection by
  $\{C_i\}_{i=1}^{2^{dk}}$ and sum over $i$, we see that
  $\sum\limits_{i=1}^{2^{dk}}(diam(C_i))^d
  =\sum\limits_{i=1}^{2^{dk}}(2^{-k}\sqrt{d})^d=\sqrt{d}^d < \infty$. Since
  $\forall \delta>0$, we can choose $k$ such that $2^{-k}\sqrt{d}<\delta$, we
  see that $m_d^*(C)=\inf \{\sum\limits_{i=1}^{\infty}(diam(F_i))^d|C \subset
  \bigcup\limits_{i=1}^{\infty}F_i$, $F_i \in P^\mathbb{X}$, and
$diam(F_i)<\delta\} \leq \sqrt{d}^d < \infty$. Since $m_d^*(C)<\infty$, we see
by theorem 1 that $dim_H(C) \leq d$. 
  
  It is also true, but harder to prove, that $m_d^*(C)>0$. This fact and
  theorem 2 together tell us that $dim_H(C) \geq d$. Therefore, $dim_H(C)=d$.
  
  By the scaling property described above, we see that for $r>0$,
  $m_d^*(rC)=r^{d}m_d^*(C)$, so that every cube in $\mathbb{R}^d$ has finite,
  positive, d-Hausdorff outer measure, and so must have Hausdorff dimension
  $d$. Moreover, since every bounded set is a subset of some cube, and since
  the d-Hausdorff outer measure is monotonic, we see that every bounded set $A$
  has $m_d^*(A)<\infty$. Therefore, every bounded set in $\mathbb{R}^d$ has
  $dim_H(A) \leq d$. The reason we cannot assume that the Hausdorff dimension
  of every set $A \subset \mathbb{R}^d$ is precisely $d$ is because it may be
  that $m_d^*(A)=0$. In this case, theorem 2 does not apply and thus cannot
  give us a lower bound on the Hausdorff dimension of the set $A$.

  
