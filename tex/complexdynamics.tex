\chapter{Complex Dynamics}
\label{chap:complex dynamics}
{Written by Ben Kushigian}
\section{The Complex Plane}
One of the most beautiful areas of mathematics is complex analysis. From 
linear algebra we know that we may treat \(\R^2\) as a vector space with
vector addition and scalar multiplication. It would be desirable to define
a multiplicative operation \(* : \R^2 \times \R^2 \rightarrow \R \times \R\)
that behaves similarly to multiplication in \(\R\). First, since \(R\) is a
subset of \(\R^2\) we would like \((x,0) * (y,0) = (xy,0)\); that is, we want
\(*\) to generalize our familiar notion of multiplication. Next, we would like
to have \(|(x_1, y_1) * (x_2, y_2)| = |(x_1, y_1)| * |(x_2, y_2)|\), where
\(|(x,y)| = \sqrt{x^2 + y^2}\) is called the {\em modulus} of a complex number.
It turns out that defining \((x_1,y_1)*(x_2,y_2) = (x_1x_2-y_1y_2, x_1y_2 + 
x_2y_1)\) gives us precisely what we wanted; elementary calculation shows that
the modulus of the product is equal to the product of the moduli and for
two real numbers \((a,0)\) and \((b,0)\) we have \((a,0)*(b,0) = (ab,0)\).\\

We denote \(R^2\) along with vector addition \(+\) and multiplication as defined
\(*\) by \(\C\) and call it the complex plane. Further we make frequent use
of the notation \(i:= (0,1)\). For a complex number \((x,y)\) we denote 

